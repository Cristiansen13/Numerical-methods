\documentclass{article}
\usepackage[margin=2cm]{geometry}
\usepackage{graphicx}

\title{ReadMe}
\author{Strejaru Mihai-Cristian}
\date{Mai 2023}
\begin{document}
\maketitle
\vspace{-6cm}
\includegraphics[width=2cm]{logo.jpg}
\includegraphics[width=2cm]{upb.png}
\vspace{5cm}
\section{Task 1}
\begin{itemize}
    \item Acest algoritm aplică Descompunerea Valorilor 
    Singulare (SVD) unei matrici foto date, reduce 
    dimensiunile acesteia prin selectarea celor mai 
    semnificative componente și reconstruiește o 
    aproximare a matricei originale.
\end{itemize}

\section{Task 2}
\begin{itemize}
    \item Algoritmul aplică PCA pentru a reduce 
    dimensiunea matricei foto și a reconstrui o 
    aproximare a acesteia, utilizând un număr specificat 
    de Componente Principale pentru 
    aproximare. Prin reducerea dimensiunilor, algoritmul
    obține o reprezentare compactă a imaginii, păstrând 
    totuși informațiile importante necesare pentru 
    reconstrucția acesteia.
\end{itemize}

\section{Task 3}
\begin{itemize}
    \item Se efectuează Analiza Componentelor Principale (PCA) 
    pentru a reduce dimensionalitatea matricei foto de 
    intrare și se reconstruiește o aproximare a matricei 
    originale prin selectarea Componentelor Principale 
    cele mai semnificative. Procesul implică calcularea 
    matricei de covarianță, găsirea vectorilor și a 
    valorilor proprii, selectarea componentelor principale
    relevante și transformarea și reconstrucția matricei 
    pentru a obține o compresie a datelor și o reprezentare 
    compactă a imaginii.
\end{itemize}

\section{Task 4}
\begin{itemize}
    \item Imaginile de antrenament și etichetele corespunzătoare 
    sunt încărcate din fișier folosind comanda load. 
    Apoi, folosesc algoritmul PCA pentru a reduce 
    dimensionalitatea datelor și pentru a extrage 
    caracteristicile relevante ale imaginilor. Aceasta 
    implică calcularea mediei fiecărei coloane și 
    normalizarea datelor, calcularea matricei de 
    covarianță, determinarea vectorilor și valorilor 
    proprii, sortarea acestora și selectarea componentelor 
    principale. Apoi, se proiectează imaginile de 
    antrenament în spațiul componentelor principale și se 
    reconstruiește imaginea de test utilizând doar un număr
    specificat de componente principale. Folosesc algoritmul
    k-nearest neighbors pentru a prezice cifra reprezentată 
    de imaginea de test, selectând cele mai apropiate k 
    imagini din setul de antrenament.
\end{itemize}

\section{Observatii}
\begin{itemize}
    \item Cu cât numarul de componente semnificative 
    incluse(k) ese mai mare, cu atat se va obtine o 
    aproximare mai precisă a matricei foto inițiale. 
    În schimb, cu cât k este mai mic, cu atât mai multă 
    informație este pierdută în procesul de reducere a 
    dimensiunilor, ceea ce poate duce la o imagine 
    generată mai comprimată și cu detalii mai puține. 
    Deci, alegerea valorii optime pentru k este importantă
    pentru a obține un echilibru între calitatea 
    aproximării și nivelul de comprimare dorit.
\end{itemize}

\end{document}